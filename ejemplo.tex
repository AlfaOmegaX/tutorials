% Hola! Bienvenido a este documento LaTeX de ejemplo. Para
% compilarlo, ejecuta `pdflatex ejemplo.latex` en tu terminal.
% Hay otros compiladores de LaTeX (xelatex, lualatex), la única
% diferencia apreciable es el soporte de algunos paquetes y de
% fuentes.

% Esto es el preámbulo. En esta parte del documento puedes cargar
% paquetes y definir nuevos comandos, entre otras cosas.

% El primer comando del preámbulo suele ser la especificación del
% tipo de documento que vas a escribir. En este caso es un artículo
% en formato A4, pero podría ser un libro, una presentación, etc.
\documentclass[a4paper]{article}

% Para escribir en español necesitarás acentos y eñes, así que
% será necesario indicar la codificación del documento:
\usepackage[utf8]{inputenc}
\usepackage[T1]{fontenc}

% Además, algunos comandos de LaTeX escriben texto en el documento,
% procuraremos que esté en español:
\usepackage[spanish,es-noquoting,es-lcroman]{babel}
\selectlanguage{spanish}

% Vamos a incluir algunos paquetes. Los paquetes básicos para
% escritura de documentos matemáticos son:
\usepackage{amsmath,amsfonts,amsthm}

% Un paquete más, para poder añadir enlaces
\usepackage{hyperref}

% Con eso será suficiente por ahora. Recuerda que si quieres que
% tu documento cuente con alguna funcionalidad fuera de lo
% habitual, suele haber un paquete de LaTeX que la define. Por
% ejemplo, tablas muy grandes, diagramas, tipografías, etc.

% Lo último que haremos en este preámbulo será declarar el título
% y el autor de nuestro documento
\title{Ejemplo de documento en LaTeX}
\author{David Charte}

% Genial. Ya estamos listos para comenzar a escribir. El contenido
% hay que englobarlo en el entorno `document` con \begin y \end:
\begin{document}

% Uou, ahora lo que escribas aquí aparecerá en tu documento al
% compilarlo, pero antes de que se nos olvide vamos a colocar
% el título que hemos definido antes:
\maketitle

% Vamos a crear una sección (título de primer nivel) para no ir
% poniendo texto suelto a lo loco
\section{Introducción}

Este documento PDF ha sido generado al compilar la \textit{demo}
\texttt{ejemplo.latex} que viene incluida en el repositorio de
preparación para primer curso. En las siguientes secciones se exploran
algunas de las opciones básicas de \LaTeX\ para formateo de texto e
inclusión de otros elementos, como ecuaciones, tablas y figuras.

\section{Formato}

Dar formato al texto en \LaTeX\ es relativamente sencillo. Basta con
usar algunos comandos usuales para \textbf{resaltar texto en negrita},
tal vez \textit{añadir algunas palabras en cursiva}, incluso
\texttt{código en monoespaciada}. Para añadir un bloque de código
podemos usar el entorno \texttt{verbatim}.

\begin{verbatim}
def count_words string
  string.split(" ").length
end
\end{verbatim}

\section{Matemáticas}

% Ey, para hacer títulos de segundo nivel podemos crear "subsecciones"
% (y claro, también hay subsubsecciones para los de tercer nivel).
\subsection{El entorno matemático}
El código que escribes en \LaTeX\ regularmente sólo acepta texto
normal, no podemos escribir ecuaciones directamente sobre él. Sin
embargo, hay comandos y entornos que activan el \textbf{entorno
  matemático}. Dentro de él, podrás dibujar símbolos matemáticos
mediante comandos especiales.

\subsection{Ecuaciones en línea}
El entorno matemático en línea (es decir, integrado en el párrafo de
texto) se activa y termina con el signo del dólar \texttt{\$}. Veamos
un ejemplo: $e^{i\tau}=1$.

\subsection{Ecuaciones en bloque}
A las ecuaciones que ocupan su propio renglón y van separadas del
texto se les suele llamar \textit{en bloque}. Se pueden englobar en el
entorno \texttt{equation} o bien entre \textbackslash[ y
\textbackslash]. La diferencia entre estas dos opciones es que la
primera numerará la ecuación
\begin{equation}
  f:\mathbb{R}^2 \rightarrow \mathbb{R}^2
\end{equation}
y la segunda no (un resultado similar se consigue con el entorno
\texttt{equation*}):
\[ f((x, y)) = (-y, x) \]

\subsection{Símbolos}

En el modo matemático podemos escribir variedad de símbolos, por
ejemplo las letras griegas $\alpha, \beta, \gamma$ y las demás. Para
una referencia más completa, visita
\href{https://en.wikibooks.org/wiki/LaTeX/Mathematics#List_of_Mathematical_Symbols}{
  este libro de Wikibooks}.

\subsection{Más ecuaciones}
\[
  \delta_{ij} =
  \begin{cases}
    1 & \mbox{ si } i = j \\
    0 & \mbox{ en otro caso}
  \end{cases}
\]

\[
  Ax = 
  \begin{pmatrix}
    \cos\alpha & -\sen\alpha & 0  \\
    \sen\alpha & \cos\alpha  & 0  \\
    0          & 0           & -1
  \end{pmatrix}
  \begin{pmatrix}
    x_1 \\
    x_2 \\
    x_3
  \end{pmatrix}
\]

           

\section{Figuras, tablas y otros}

\end{document}
